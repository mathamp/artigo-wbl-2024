\section{Respostas aos Revisores}

\begin{verbatim}
Review 2

*** Relevância Avalie a importância do tema, das questões abordadas 
e dos resultados do trabalho. Relacione esta importância com o escopo 
do evento. Caso o trabalho não seja bem avaliado neste item, apresente 
algumas sugestões aos autores de como torná-lo mais relevante.
   1: Baixa
   2: Moderada
   3: Alta

  Evaluation = 2: Moderada

*** Originalidade Avalie a originalidade do trabalho, comparando 
com os trabalhos já existentes. Avalie se o trabalho apresenta novos 
resultados ou novas observações relevantes sobre um tema já tratado 
em outros artigos. Inclua nos comentários para o autor as obras 
relacionadas ao texto que não foram citadas.
   1: Nenhuma originalidade
   2: Pouco original
   3: Razoavelmente original
   4: Muito original

  Evaluation = 2: Pouco original

*** Mérito técnico Avalie o mérito do trabalho proposto, analisando 
a qualidade da sua ideia central e a profundidade do autor na 
compreensão do tema, dos problemas e das soluções apresentadas. Nos 
comentários aos autores, inclua sugestões que possam melhorar a 
qualidade e a profundidade do trabalho.
   1: Tecnicamente fraco e contribuições fracas
   2: Tecnicamente fraco e com contribuições marginais
   3: Tecnicamente consistente com contribuições marginais
   4: Tecnicamente consistente com contribuições

  Evaluation = 3: Tecnicamente consistente com contribuições marginais

*** Organização e clareza do texto Avalie a estrutura do texto, a 
sua legibilidade e a qualidade didática da redação. Verifique se a 
formatação está adequada e se as figuras estão bem relacionadas no 
texto.
   1: Inaceitável
   2: Pobre
   3: Média
   4: Boa
   5: Excelente

  Evaluation = 4: Boa

*** Referências bibliográficas Em relação à qualidade das referências 
bibliográficas, qual das opções mais se adequa para caracterizar o 
artigo?
   1: Inaceitável
   2: Pobre
   3: Médio
   4: Bom
   5: Excelente

  Evaluation = 3: Médio

*** Recomendação geral Qual a sua recomendação geral sobre o artigo?
   1: Tenho fortes argumentos contra a aceitação do artigo
   2: Prefiro que o artigo seja rejeitado, mas não argumentarei se o 
      mesmo for aceito pelos outros revisores
   3: Prefiro que o artigo seja aceito, mas não argumentarei se o 
      mesmo for rejeitado pelos outros revisores
   4: Tenho fortes argumentos para aceitar o artigo

  Evaluation = 3: Prefiro que o artigo seja aceito, mas não 
  argumentarei se o mesmo for rejeitado pelos outros revisores

*** Parecer Caso você concorde que o artigo deva ser aprovado, 
apresente aos autores informações que você julga relevantes para 
gerar a versão final do artigo a ser incluída nos anais do WBL. 
Se você entretanto achar que o artigo não deve ser selecionado, 
apresente no texto abaixo os motivos para isso de forma que os 
autores tenham a oportunidade de melhorar o trabalho para futuras 
submissões.

  Evaluation = Os autores realizaram a modelagem de uma rede IOT 
  simples, segundo uma arquitetura estabelecida na literatura 
  (Al Farooq et al. 2019) e a implementaram no framework de 
  reescrita Maude visando realizar a detecção de conflitos 
  automática. O trabalho está bem escrito apesar de pequenos 
  erros que indico abaixo. Apesar da simplicidade do modelo, 
  os autores experimentaram problemas que aparecerão em redes 
  reais, maiores e mais complexas e tiveram que lidar com formas 
  de resolver e contornar essas situações. A principal contribuição 
  do trabalho é prática, ou seja, a especificação de uma rede IoT 
  em Maude, com certas decisões de design (descritas na seção 5 do 
  trabalho), que podem ser utilizadas em outros contextos por outros 
  pesquisadores. Acredito que esse tipo de trabalho e contribuição 
  se alinhe com os objetivos do WBL e, portanto, recomendo a aceitação.

Pequenos erros encontrados e sugestões de melhoria:

Pág. 1: Sistemas IoT estão cada vez mais utilizados no mundo... -&gt; 
Sistemas IoT são cada vez mais utilizados no mundo...
\end{verbatim}

{\bf R:} \matheus{Done.}

\begin{verbatim}
Pág. 3: Ao explicar os 2 comandos Maude para detecção de conflito, 
no segundo caso (search) cabe uma discussão do possível crescimento 
exponencial do grafo de busca em casos reais.
\end{verbatim}

{\bf R:} \daniel{TODO}

\begin{verbatim}
Pág. 3: O modelo é baseado no modelo proposto... -&gt; sugiro evitar 
a duplicidade da palavra modelo.
\end{verbatim}

{\bf R:} \daniel{Done.}

\begin{verbatim}
Pág. 3/4: um grupo de componentes estão configurados -&gt; um grupo 
de componentes está configurado (2x)
\end{verbatim}

{\bf R:} \daniel{Done.}

\begin{verbatim}
Pág. 5:  ...intencionalmente erroneamente configurados... -&gt; a 
frase ficou estranha.
\end{verbatim}

{\bf R:} \daniel{Done.}

\begin{verbatim}

Review 3

*** Relevância Avalie a importância do tema, das questões abordadas 
e dos resultados do trabalho. Relacione esta importância com o escopo 
do evento. Caso o trabalho não seja bem avaliado neste item, apresente 
algumas sugestões aos autores de como torná-lo mais relevante.
   1: Baixa
   2: Moderada
   3: Alta

  Evaluation = 3: Alta

*** Originalidade Avalie a originalidade do trabalho, comparando 
com os trabalhos já existentes. Avalie se o trabalho apresenta 
novos resultados ou novas observações relevantes sobre um tema 
já tratado em outros artigos. Inclua nos comentários para o autor 
as obras relacionadas ao texto que não foram citadas.
   1: Nenhuma originalidade
   2: Pouco original
   3: Razoavelmente original
   4: Muito original

  Evaluation = 2: Pouco original

*** Mérito técnico Avalie o mérito do trabalho proposto, analisando 
a qualidade da sua ideia central e a profundidade do autor na 
compreensão do tema, dos problemas e das soluções apresentadas. 
Nos comentários aos autores, inclua sugestões que possam melhorar 
a qualidade e a profundidade do trabalho.
   1: Tecnicamente fraco e contribuições fracas
   2: Tecnicamente fraco e com contribuições marginais
   3: Tecnicamente consistente com contribuições marginais
   4: Tecnicamente consistente com contribuições

  Evaluation = 3: Tecnicamente consistente com contribuições 
  marginais

*** Organização e clareza do texto Avalie a estrutura do texto, 
a sua legibilidade e a qualidade didática da redação. Verifique 
se a formatação está adequada e se as figuras estão bem 
relacionadas no texto.
   1: Inaceitável
   2: Pobre
   3: Média
   4: Boa
   5: Excelente

  Evaluation = 4: Boa

*** Referências bibliográficas Em relação à qualidade das 
referências bibliográficas, qual das opções mais se adequa 
para caracterizar o artigo?
   1: Inaceitável
   2: Pobre
   3: Médio
   4: Bom
   5: Excelente

  Evaluation = 4: Bom

*** Recomendação geral Qual a sua recomendação geral sobre 
o artigo?
   1: Tenho fortes argumentos contra a aceitação do artigo
   2: Prefiro que o artigo seja rejeitado, mas não argumentarei 
      se o mesmo for aceito pelos outros revisores
   3: Prefiro que o artigo seja aceito, mas não argumentarei se 
      o mesmo for rejeitado pelos outros revisores
   4: Tenho fortes argumentos para aceitar o artigo

  Evaluation = 3: Prefiro que o artigo seja aceito, mas não 
  argumentarei se o mesmo for rejeitado pelos outros revisores

*** Parecer Caso você concorde que o artigo deva ser aprovado, 
apresente aos autores informações que você julga relevantes para 
gerar a versão final do artigo a ser incluída nos anais do WBL. 
Se você entretanto achar que o artigo não deve ser selecionado, 
apresente no texto abaixo os motivos para isso de forma que os 
autores tenham a oportunidade de melhorar o trabalho para 
futuras submissões.

  Evaluation = O presente trabalho discute uma aplicação da 
  linguagem/sistema MAUDE para especificação e verificação 
  de propriedades em um ambiente IoT simulado.
Enquanto o texto é majoritariamente claro em sua discussão, 
os autores por vezes pecam por brevidade ou superficialidade 
na apresentação. Isso é visível em seções, como a 2, em que 
sistemas de re-escrita são pouco formalizados - o que dificulta 
o entendimento dos resultados dos autores por audiências não 
familiarizadas com o tópico.
\end{verbatim}

{\bf R:}

\begin{verbatim}
Na discussão do estudo de caso, em que os autores argumentam 
sua escolha de simplificar a modelagem para diminuir o espaço 
de busca do raciocinador, seria interessante discutir a 
complexidade especifica e a diferença de expressividade entre 
as duas modelagens.
\end{verbatim}

{\bf R:}
