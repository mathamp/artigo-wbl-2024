\section{Conclusões} \label{sec:chap7}
Apresentamos um modelo para detecção de conflitos e realizamos a sua especificação no Maude, que reportou corretamente os conflitos nos casos esperados e até mesmo em casos inesperados.

A especificação da concorrência deste sistema com regras de reescrita foi bastante natural, sendo considerado um aspecto positivo. Entretanto, como a configuração de cada componente é modelada como uma regra de reescrita, para testar novas configurações, é necessário escrever novas regras, que devido ao seu tamanho e complexidade, compromete a escalabilidade da abordagem e acaba dificultando o processo iterativo de novas configurações.

Certos aspectos da especificação do modelo tiveram que ser alterados para que o Maude possa fazer buscas em um tempo razoável. Essas alterações tem como principal objetivo reduzir o tamanho do espaço de busca.

Como sugestões de trabalhos futuros, os seguintes aspectos desse trabalho podem ser explorados:
\begin{itemize}
  \item Expandir o domínio de estados dos objetos IoT de valores binários para valores discretos. Isto permitiria um controle mais refinado dos objetos, como controlar a temperatura do ar condicionado, ou a luminosidade das lâmpadas.
  \item Explorar a possibilidade de utilizar mais condições relacionais. Atualmente os eventos são disparados apenas se uma variável de ambiente atinge um valor exato.
  \item Este trabalho detecta somente conflitos diretos. Na literatura existem outros tipos de conflitos: indiretos e implícitos. Conflitos indiretos são possíveis de serem detectados formalmente, mas estes devem ser identificados de antemão, utilizando ferramentas de análise estatísticas. Por exemplo, ligar o ar condicionado e abrir a janela não são ações contraditórias, mas podem ser detrimentais para o ambiente. Para mais informações, veja \cite{ConflictDetectionORAN}.
  \item Este trabalho aborda somente a detecção de conflitos. Mitigação automática de conflitos é uma possível extensão do trabalho que pode ser explorada.
  \item Identificar se existem regras de reescrita que podem ser trocadas por equações. Isto reduziria o crescimento do grafo de busca e permitiria buscas em configurações mais complexas.
  \item No Maude, é possível definir estratégias para aplicação de regras, que impediria a exploração de estados indesejáveis. O uso de estratégias pode ser usado para minimizar o crescimento da árvore de busca. \cite{MaudeExplanation}
\end{itemize}