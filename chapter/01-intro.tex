\section{Introdução} \label{sec:chap1}

Sistemas IoT são cada vez mais utilizados no mundo, para diversas finalidades como casas inteligentes ou até mesmo cidades inteligentes, com dispositivos conectados remotamente e que conseguem medir e gerenciar algum aspecto do seu ambiente \cite{ConflictDetectionSurvey}. Estes dispositivos podem ser controlados remotamente por algum aplicativo, ou configurados para agirem automaticamente baseado em certos gatilhos.

Montar uma rede IoT pode ser um grande desafio, pois há aspectos importantes que devem ser considerados para garantir que os dispositivos estão agindo corretamente, que as regras configuradas garantam a integridade e a segurança do sistema \cite{ConflictDetectionFuture}.

Este trabalho apresenta uma modelagem para detecção de conflitos em redes distribuídas utilizando o Maude, um framework de Lógica de Reescrita de alta performance. A seção 2 deste trabalho contém uma introdução à Lógica de Reescrita e ao Maude. A seção 3 contém uma descrição do modelo da nossa rede e seus componentes. A seção 4 contém uma breve descrição para o nosso caso de estudo. A seção 5 explica como foi realizada a especificação no Maude. A seção 6 contém os resultados do nosso trabalho e possíveis trabalhos futuros. A seção 7 contém a nossa conclusão sobre este trabalho.

A especificação completa do modelo no Maude pode ser encontrada no endereço: \texttt{https://github.com/mathamp/conflict-detection-maude}.